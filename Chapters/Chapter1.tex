\chapter{总述} % Main chapter title

\label{Chapter1} % Change X to a consecutive number; for referencing this chapter elsewhere, use \ref{ChapterX}

硬件资源,包括计算机处理器计算资源、内存容量、存储设备容量与IO、网络带宽等,是
处理各类计算任务的核心。
随着时代的进步,工业界的需求对硬件资源不断提出新的要求,驱动着硬件技术的发展与
进步。
其中,硬件资源分离作为本文讨论的主题,是一种新兴的提高硬件资源利用率的研究方向
与技术类别。
本章将从计算的发展开始介绍,列举当前大规模计算下的挑战,并阐述为何硬件资源分离
成为了当前备受关注的解决方案。

%----------------------------------------------------------------------------------------
\section{计算的发展}
%----------------------------------------------------------------------------------------

%--------------------------------------
\subsection{计算能力的增加}
%--------------------------------------

在过去几十年,计算机计算能力不断提升。
从最早的电子管、晶体管,到现在普遍使用的多核处理器,计算翻倍的摩尔定律几乎从未失
效。
现在,英特尔最新的第九代酷睿处理器,处理器频率最高能达到5GHz
\cite{intel2018corei9}。
除此之外,计算机内存技术、存储技术、网络技术等也都在不断地发展。
内存除了容量与速度不断增大之外,还出现了许多新的特性,比如非易失性
(non-volatile)\cite{wikipedia2019nvm}。
存储技术的发展也与内存相似,随着新的介质提出与工艺改进,在容量与速度上存储技术
稳步提高。与此同时,在不同介质下,存储设备还会有细微的性能特性,比如随机与顺序
访问的性能差异。
其他支撑技术也在近年来得到不断发展。

%--------------------------------------
\subsection{计算需求的发展}
%--------------------------------------

与计算能力增加同时进行的,还有计算需求也在不断的变多、变高。
最初计算机仅用于科学计算,且数据量在现在看来是非常小的。
随着计算机走向民用与新世纪互联网的普及,各式各样的计算任务被提出,多种计算需求
同时出现。
现在,在不同的场景下,计算任务有着截然不同的计算需求,并且可以被分入多种类别。
有关注低延时的计算任务,也有关注可靠性的计算任务,有需要读取大量数据的计算任务
,也有将大部分时间用在计算上的计算任务。

%--------------------------------------
\subsection{计算方式的变化}
%--------------------------------------

从单机计算,到小型分布式系统,再到支撑与计算平台的数据中心,计算的方式发生了巨
大的转换。
当计算需求提高,计算任务的弹性增加,往将的计算任务用在单独特用的计算机的方式不
再流行。
通过建立数据中心,运行着多个分布式计算系统,将各式各样计算任务收集在一起,协调
调度并执行,计算资源得到更有效地利用,而单一使用者空闲导致资源限制的问题得到缓
解。

%----------------------------------------------------------------------------------------
\section{大规模计算下的资源挑战}
%----------------------------------------------------------------------------------------

为了处理大规模的计算,现有的系统往往面临多多方面的挑战,包括:
\begin{itemize}
    \item
    最大化利用计算资源;
    \item
    克服节点间带宽的限制;
    \item 
    低成本运维海量硬件;
    \item
    在不可靠硬件基础上提供高可靠的计算能力和存储能力;
    \item
    提供高可用服务等。
\end{itemize}

其中,资源利用率问题是我们在本文关注的重点。

数据中心存放管理这大量的计算资源,并且可以动态地分配给不同的资源消费者,及各类
计算任务或者进一步的资源管理者\cite{zhang2010cloud}。
早期常用的方法是直接通过搭建在多台机器上的分布式系统,直接将任务,显式地分配给
某台机器进行执行。
为了便于管理与调度,现在以虚拟机与容器为单位的资源管理变得流行。
无论是什么方式,都存在着各种任务对资源的需求不一致,难以统一协调,最终导致资源
分配不足或者资源浪费的问题。
在商用计算平台上,因为出服务的稳定性要求,资源超供导致的闲置问题更为突出。
如何在保证服务的质量前提下,如何更高地提高资源利用率成为了当先热门的商业痛点与
研究方向。

%----------------------------------------------------------------------------------------
\section{硬件资源分离}
%----------------------------------------------------------------------------------------

硬件资源分离是一类关注在硬件层面,尝试为了提高资源利用率的方法。
其核心思想是,不将硬件资源与本地计算机强绑定在一起。
允许通过特定的机制,将不同机器的某项资源同时服务于某项计算任务,从而提高整体的
资源利用率,加速计算的完成。

在本文中,我们将从关键的两类资源——内存与存储的资源分类入手,介绍针对单一资源的
分离是如何完成的。
随后,我们将从系统层面的资源分离,即如何将多种资源在统一的框架下进行分离使用进
行介绍。
每部分内容都会配合相关的最新研究成果作为案例分析。