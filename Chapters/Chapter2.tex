% Chapter Template

\chapter{内存分离} % Main chapter title

\label{Chapter2} % Change X to a consecutive number; for referencing this chapter elsewhere, use \ref{ChapterX}

% 论点:
% 1. 阐述内存分离技术的重要意义
% 2. 阐述内存分离技术目前的研究现状,相当于related work
% 3. 以INFINISWAP为例来最先进内存分离技术。
% 4. 自己的启发性思考
%

内存资源在数据中心中相对昂贵而且稀少的资源,如何利用好内存资源一直都是学术界以及工业界的一个研究重点。
其中,一个主要研究方向是内存分离(memory disaggregation)。
将内存资源从一体化服务器(monolithic server)中解耦出来,服务器上的应用能同时访问“本地”的和“远端”的内存,看上去就像是访问一块内存一样,这样的技术被称为内存分离(memory disaggregation)。
通过内存分离,应用程序能使用到更多的内存资源,也减少了内存利用率不高而导致的资源、能源浪费的情况。
本章节将主要通过介绍INFINISWAP系统,来介绍内存分离的一个具体实现,并通过这个案例来理解内存分离的作用与意义。

在本章的开始,我们先介绍内存分离技术的研究背景,从而了解数据中心对内存资源的需求以及内存分离技术出现的缘由。
再介绍内存分离的研究现状,介绍内存分离不同的研究方向以及其对应的技术。
接着,我们着重介绍再内存分离技术领域比较先进的INFINISWAP系统\cite{gu2017efficient}。
最后我们对内存分离技术做一个总结,并提出一些我们的问题和思考。

%----------------------------------------------------------------------------------------
%	SECTION 1
%----------------------------------------------------------------------------------------

\section{背景介绍}

\subsection{问题来源}
\paragraph{内存不足} 
内存不足的现象是自内存设备出现以来就已经存在的问题。
在过去,解决内存不足问题的目标仅仅是说让程序在内存不足的时候也能正确运行,因此产生了虚拟内存(virtual memory)、换页(paging)等技术来缓解内存不足的状况,本质是通过硬盘来暂存内存数据,这样的方法虽然解决了内存不足的问题,但性能却很差。
现在,虽然单机上的内存的容量已经增大了很多,但内存不足的情况依旧存在。
因为应用程序对于内存的需求也越来越大了,而换页需要读写硬盘,性能很差,并不能满足应用的需求。
内存不足的问题导致的是应用程序的性能下降。

\paragraph{内存使用不均衡} 
除了内存不足的问题之外,在数据中心集群内部,内存使用还有不均衡的问题。
对于单台服务器来说,它将会运行各式各样的应用程序,而且会同时运行多个不同的应用程序。
每个应用程序对内存的使用需求并不一致,但是每台服务器的可用的内存容量却是固定的。
因此,必然会出现某些服务器在某段时间内,内存并不会完全被使用,甚至利用率很低。
结合上面上面的内存不足问题,在集群中,就可能出现这样的现象:一部分机器的内存不足,另一部分机器的内存过剩,这就是内存使用不均衡的现象。
内存使用不均衡的问题导致的是资源浪费。\\

无论是内存不足的问题还是内存使用不均衡的问题,它的根本原因是传统的一体化服务器的架构给内存的部署、分配和使用所带来的限制。
一方面,单机上内存资源在服务器部署之后就已经基本上是确定不变的了,服务器部署运行之后再去拓展或减少内存资源是一件非常麻烦的事情。
另一方面,单台服务器所能支持的最大内存有限,而且机器之间不能直接使用对方的内存,这必然不能满足应用程序日益增长的内存需要。

\subsection{问题展示}
首先,为了说明内存不足而导致换页的不利影响,有相关研究人员做了一些测试\cite{gu2017efficient}。
他们选择了三种不同的内存应用程序进行测试:(i)运行在VoltDB内存数据库上的TPC-C基准测试程序;
(ii)运行在Memcached键值仓储中的模拟facebook工作负载的程序;
(iii)运行在PowerGraph的TunkRank算法,使用的数据集是Twitter。

\begin{figure}
\centering
\includegraphics[scale=0.45]{Figures/memory/memory_motivation1.png}
\decoRule
\caption{在不同内存容量下应用程序的性能\cite{gu2017efficient}。}
\label{fig:memory_motivation1}
\end{figure}

为了避免由外部因素造成的性能影响,测试只关注应用程序在单机下的性能(图~\ref{fig:memory_motivation1})。
结果显示由于内存不足所导致的换页确实对应用程序产生重大的、非线性的、急速下降的性能影响。
另外,换页会导致非常明显的尾延迟现象。
以上的现象都表明,内存不足是一个非常重要的问题,进行内存分离的研究是十分有必要的。

\begin{figure}
\centering
\includegraphics[scale=0.6]{Figures/memory/memory_motivation2.png}
\decoRule
\caption{Facebook和Google的两个机器集群中存在的内存使用不均衡现象\cite{gu2017efficient}。}
\label{fig:memory_motivation2}
\end{figure}

其次,不同机器中的内存使用也存在着不均衡的现象,这将导致资源浪费。
有相关的研究人员统计了Google和Facebook的两个实际运行的集群中机器内存使用情况的一些数据(图~\ref{fig:memory_motivation2})。
通过记录和计算10秒内前99\%的机器的平均内存使用率与所有机器平均使用率的比值,来表示内存使用的不均衡性。
结果显示,集群中有超过一半的内存因资源利用不均衡而导致其未被使用到的。
这样的资源利用不均衡也验证了我们的说法。


%----------------------------------------------------------------------------------------
%	SECTION 2
%----------------------------------------------------------------------------------------

\section{当前内存分离技术的研究现状}

很长一段时间以来,数据中心都在使用着一体化服务器的架构,使得大多数内存分离技术都是基于这种架构去设计并实现的,如分布式共享内存(distributed share memory)技术和远程换页(remote paging)技术。
这些方法考虑的是易用性、可行性,不需要对现有服务器架构设计进行太大的修改。

内存分离技术有多种的划分方法。
LegoOS\cite{shan2018legoos}的研究人员按照软硬件的划分将内存分离技术分成两大类,一类是硬件辅助的内存分离技术,指的是设计一些专用的小硬件来辅助和加速内存分离;
另一类是纯软件实现的内存分离技术,只需要现有的硬件设备来实现内存分离。
而INFINISWAP的研究人员则是按照实现方法来划分,将内存分离技术主要分为远程内存换页技术(Remote Memory Paging)和分布式共享内存(Distributed Shared Memory)两大类。

我们认为INFINISWAP的划分思路更加贴合内存分离的主题,我们在此基础上做一些补充,将内存分离技术可以分为如下几类:

\subsection{应用层面的分布式内存管理}

将应用部署到多台机器上,每台机器上的应用主要去管理和使用本地的内存,它们之间通过网络获取和共享各自的内存数据,这是应用层面的分布式内存管理。
其与其它内存分离技术的最大区别是,应用程序是知晓自己是处在一个分布式集群环境中的,开发人员要根据应用程序的具体情况设计内存的管理以及机器之间数据交换的方法和协议。
通常,内存密集型的一些分布式应用通常都有比较好的内存管理和交换方法,比如内存数据库、内存图计算等等,相关的系统有:DrTM\cite{wei2015fast}和Wukong\cite{shi2016fast}。

应用层面的分布式内存管理解决了单台服务器内存不足的问题,而且它灵活度高,开发人员能够根据应用程序的具体情况去设计高性能的内存管理方法和数据交换方法。
但是,这种方法也有缺点,分布式应用编程开发和调试很具有难度,而且它也并没有解决由应用程序之间内存需求不一致而导致的内存使用不均衡问题。

\subsection{远程内存换页技术}
远程内存换页技术指的是,通过网络将本地的内存页换入换出到远端机器的内存中,而不是传统那样换到本地的硬盘\cite{chen2008transparent,newhall2003nswap,flouris1999network}。
在过去,远程内存换页技术却受限于低缓的网络传输和过多的CPU的额外开销。
而现在,由于高速网络如RDMA技术的出现,远程内存换页技术的性能有了很大的提升,目前,在这一领域最先进的工作和成果是INFINISWAP\cite{gu2017efficient}。

远程换页技术对应用程序透明,单机的应用程序可以不需要进行修改便可以使用远程换页技术,因而通用性很好。
除了上面所提到的网络传输和CPU开销问题,通常的远程内存换页技术还需要中央协同器来进行页淘汰和负载均衡等工作。

\subsection{分布式共享内存}
分布式共享内存系统将集群所有机器的内存在逻辑上组织成一个全局的地址空间,供应用程序去使用,集群内部任何一台机器上的应用程序都可以对这一地址空间进行读写\cite{carter1991implementation,li1989memory,nitzberg1991distributed}。
过去的分布式共享内存技术因维护一致性而产生过多的通信开销。
现在分布式共享内存技术也使用了一些新的硬件,比如PGAS和RDMA,虽然减少了维护一致性的开销,但需要开发人员重写应用程序来使用它的接口。

分布式共享内存同样是对应用程序透明,通用性很好。
但它需要维护一致性,而且代价比较大,因而可拓展性可能会受到一定的限制。


%----------------------------------------------------------------------------------------
%	SECTION 3
%----------------------------------------------------------------------------------------

\section{案例:基于RDMA高速网络的内存分离技术——INFINISWAP}

INFINISWAP是针对RDMA高速网络设计的一个远程换页系统,通过将集群中每台机器的交换分区划分成大块去管理,适时地去收集机器中未使用的内存,并把这些内存暴露给应用程序使用。
整个过程对于应用程序来说是透明不可见的,因而应用程序不需要做任何修改。
在接下来的小节中,我们将系统设计和性能测试来对INFINISWAP系统进行介绍。

\subsection{设计目标}
INFINISWAP的主要设计目标是高效地将集群中所有机器的内存都暴露给应用程序去使用,不需要对应用程序或者操作系统做任何修改。
系统必须具备可扩展性、容错性和透明性,同时做好隔离,使用远程机器上的内存不能打扰到远程机器上应用的性能。

\begin{figure}
\centering
\includegraphics[scale=0.5]{Figures/memory/infiniswap_architecture.png}
\decoRule
\caption{INFINISWAP系统架构。\cite{gu2017efficient}}
\label{fig:infiniswap_architecture}
\end{figure}

\subsection{架构概述}
INFINISWAP包含了两个主要的组件——INFINISWAP块设备和INFINISWAP守护进程。
每台机器上都包含着这两个组件,因此每台机器的角色相同,从而实现了一个去中心化的系统(图~\ref{fig:infiniswap_architecture})。

\subsubsection{INFINISWAP块设备}
INFINISWAP块设备向虚拟内存管理器提供常规的块设备IO接口,让虚拟内存管理器将这个设备当成是一个交换分区。
当出现换页的时候,INFINISWAP块设备便可以透明地通过RDMA操作向其它机器中读取内存或写入内存。

INFINISWAP将每台机器上的内存地址空间划分成固定大小的大块(slab),大块是INFINISWAP内存映射和负载均衡的基本单位,它由很多内存页所组成。
以大块为单位进行内存映射处理的原因尽量减少远端机器上CPU的参与,减少对远端机器上应用程序的打扰,同时提升远程换页的速度。
当大块映射完毕之后,INFINISWAP块设备便可以通过RDMA读写操作来进行换页,换页的基本单位仍然是内存页,而不是大块。

\subsubsection{INFINISWAP守护进程}
INFINISWAP守护进程是运行在用户态的一个程序,仅仅参与控制层面(control plane)的活动。
它仅仅负责处理其它机器发过来的大块映射请求以及预先分配相应的内存空间。
真正的对数据层面(data plane)的操作则是通过RDMA请求然后由网卡去执行,并不会打扰目标机器的CPU。

\subsection{设计细节与实现}

\subsubsection{透明性和隔离性设计}
\paragraph{透明性}
INFINISWAP提供与传统块设备一样的接口和语义,应用程序不需要做任何修改便可以使用到更多的内存,它也不知道自己使用的是本地内存还是远程内存。
这样的透明性使得INFINISWAP系统具有很好的兼容性与通用性,能在现有的数据中心快速通入使用,不需要增加新的硬件。

\paragraph{隔离性}
INFINISWAP使用RDMA高速网络来传输内存页数据,不需要打扰远端CPU的执行。
对远程CPU的打扰主要在于大块的映射,但大块的大小是比内存页要大很多的,这样的一个设计也为了减少对远端机器CPU的打扰次数。

\subsubsection{容错性设计}
INFINISWAP对应用程序是透明的,应用程序认为远端机器上面的内存也是存在于本地的。
因此,INFINISWAP需要提供与应用程序只使用本地内存一样的语义,即当远端机器出现错误时,不能影响到本地应用的运行。

容错的方案是在进行RDMA远程换页的同时,异步地将内存也写入本地硬盘中做为备份。
由于写备份是异步的,并不在关键路径上,因此也不会对性能产生很大的影响。

实现这样的容错方案要处理一些边界问题,最重要的边界问题是处理\emph{read-after-write}的情况。
即当内存页已经写到远端的内存但还没有写入本地硬盘中,在这时候远端机器发生了崩溃。
如果在这同时应用程序又要读取这块内存页,块设备便需要从硬盘的写队列中读取内存页,然后返回给应用程序。

\begin{figure}
\centering
\includegraphics[scale=0.3]{Figures/memory/infiniswap_slab_mapping.png}
\decoRule
\caption{INFINISWAP块映射策略\cite{gu2017efficient}。}
\label{fig:infiniswap_slab_mapping}
\end{figure}

\begin{figure}
\centering
\includegraphics[scale=0.3]{Figures/memory/infiniswap_slab_eviction.png}
\decoRule
\caption{INFINISWAP块淘汰策略\cite{gu2017efficient}。}
\label{fig:infiniswap_slab_eviction}
\end{figure}

\subsubsection{可拓展性设计}
INFINISWAP并没有一个中心化的设计,避免由中央的协同器成为系统的瓶颈而导致系统不具有可拓展性。
去中心化的设计虽然能够使系统具备一定的可拓展性,但缺少中央协同器来收集全局的信息,这会带来很多问题。

其中关键的问题,如何制定好的大块映射策略来使得集群中每台机器的内存使用情况都趋于均衡。
INFINISWAP使用了\emph{power-of-choices}的技术来作为块映射以及块淘汰的策略。
比如\emph{power-of-two}的块映射策略,会先随机地选择两台远程机器,比较两台机器的内存使用情况,然后选择内存使用较少的一台机器作为目标机器来进行块映射。
类似地,块淘汰策略也是使用了\emph{power-of-choices}的方法。

\subsection{系统评测}
测试使用32台机器的集群,机器之间用56Gbps的Infiniswap网卡连接。
每台机器有2个NUMA节点和64GB的内存,每个NUMA节点有8个物理CPU,一共32个vCPU。

\begin{figure}
\centering
\includegraphics[scale=0.4]{Figures/memory/infiniswap_evaluation1.png}
\decoRule
\caption{应用程序的性能\cite{gu2017efficient}。}
\label{fig:infiniswap_evaluation1}
\end{figure}

\paragraph{应用程序的性能}
用于测试的应用程序有VoltDB、Memcached、PowerGraph和GraphX。
测试分别测了单机内存能满足应用程序100\%的内存需求、能满足应用程序50\%的内存需求和满足应用程序50\%的内存需求并使用INFINISWAP系统三种情况下应用程序的性能(图~\ref{fig:infiniswap_evaluation1})。
最终的测试结果表明,在单机内存只能满足应用程序50\%的内存需求时,使用INFINISWAP能将应用程序的性能提升2到16倍。

\paragraph{集群的内存使用情况}
测试创建了90个容器运行在集群上,每个容器使用不同的内存需求配置。
测试的结果显示集群的平均内存利用率从40.8\%提升到了60\%,提升了1.47倍。
而内存的不均衡现象也有所缓解(图~\ref{fig:infiniswap_evaluation2})。

\begin{figure}
\centering
\includegraphics[scale=0.3]{Figures/memory/infiniswap_evaluation2.png}
\decoRule
\caption{集群的内存使用情况\cite{gu2017efficient}。}
\label{fig:infiniswap_evaluation2}
\end{figure}

\subsection{案例小结}
案例介绍了在内存分离领域最先进的技术INFINISWAP。
它通过RDMA技术解决了传统的远程内存换页技术所存在的网络通信慢和过多的CPU额外开销的问题。
去除了中央协同的模式,从而实现了比较好的可拓展性,还增加了容错处理。
它在一定的程度上解决了数据中心当前存在的内存不足以及内存使用不均衡问题。

%----------------------------------------------------------------------------------------
%	SECTION 4
%----------------------------------------------------------------------------------------

\section{本章小结}

本章我们探讨了内存分离这一主题,介绍了内存分离的背景、研究现状和技术案例,说明了当前数据中心存在的内存不足以及内存使用不均衡的问题,而内存分离则是解决这些问题的主流方法。

硬件的发展正不断推动着内存分离技术的发展。
实现内存分离有很多种技术方案,但无论哪种方案,都需要依靠网络来进行数据传输。
传统的内存分离技术最大的限制是网络传输速度的限制。
但现在,越来越快的一些网络技术如RDMA的出现,使得内存分离真正变得更加切实可行。

我们认为下一步的研究方向是从架构层面去思考如何去做内存分离。
内存不足以及内存使用不均衡的根本原因是一体化服务器的架构,基于这样的架构去实现的内存分离只是缓解问题,并不能从根本上去解决问题。
在后面的章节中,我们将介绍LegoOS,看看如何从根本上去解决内存的相关问题。
