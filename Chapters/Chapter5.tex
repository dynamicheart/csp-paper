\chapter{总结} % Main chapter title

\label{Chapter5} % Change X to a consecutive number; for referencing this chapter elsewhere, use \ref{ChapterX}

硬件资源分离是未来趋势。
目前,处理器的摩尔定律正面临失效的挑战,而随着工业物联网、区块链技术与人工智能
的普及,逐渐增加并复杂化的计算需求却对计算资源提出越来越高的要求。
另一方面,大规模数据中心中存在的资源利用率不理想问题与扩展性挑战却使得计算能力
并不能有效地转化为完成的计算任务。
为解决这一矛盾,最直接有效地方法便是将硬件资源结偶,不局限与单台机器,允许其被
动态地分配给不同的计算任务,及所谓的硬件资源分离。

单纯的将硬件资源分离管理并不足够。
随着硬件资源间的结偶与新硬件技术的发展,以往的假设与顾虑往往不复存在,我们需要
针对性的进行分析与重新设计,才能真正地发掘出硬件资源分离的潜力。
同时,为了将资源发挥到极致,在分离基础上,我们也应该进一步考虑如何设计方案,使
上层软件能够更好地利用新的硬件资源模型。

我们相信并期待硬件资源分离技术被广泛使用在大型数据中心,产生重要的影响。

章节\ref{Chapter1}从计算力发展与计算需求变化的角度,阐述了大规模计算对硬件资源
利用率提出的新要求,介绍了硬件资源分离这一研究方向的背景。

章节\ref{Chapter2}对内存资源分离技术进行了介绍。
我们介绍了内存资源分离的问题现状与提高内存资源利用率的常见分离技术设计与实现,
并以INFINITESWAP为案例,详细介绍了在高速网络下,内存资源分离的挑战、解决方案与
性能评估。

章节\ref{Chapter3}对存储资源分离技术进行了介绍。
我们介绍了存储技术的发展,在当前面临的挑战,介绍并对比了目前常见的解决方案。
以当前普遍使用的闪存介质为例,我们详细介绍了其中一种解决方案的设计挑战、设计细
节与性能评估。

章节\ref{Chapter4}对系统性的硬件资源分离技术进行了介绍,本节不再关注特定一种资
源,而是如何在统一的框架下将各类资源进行分离,从而提高整体的利用率。
我们介绍了多种实现这一目标的技术与系统,并以LegoOS为案例,详细介绍分析了其设计
原则与实现细节。